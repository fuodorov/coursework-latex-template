\usepackage{amsmath,amssymb,amsfonts}%математика
\usepackage{xcolor}%цвета
\usepackage{lipsum} % for dummy text only
\usepackage[
    colorlinks = true,
    linkcolor = black,
    urlcolor  = black,
    citecolor = black,
    anchorcolor = black,
]{hyperref}% \href ссылки
\usepackage{cite}%для ссылок в библиографии с объединением
\usepackage{tabularx}%окружение таблиц tabularx
\usepackage{multirow} %объединение строк в таблицах
\usepackage{booktabs}%для всяких украшательств в таблицах (\toprule,\midrule ...)
\usepackage{float} %работа с ``плавающими'' объектами [H]
\usepackage{graphicx}% графика
\graphicspath{{figures/}}
\usepackage{wrapfig}%картинки с обтеканием

%\usepackage{minted}%листинг кода

%мощный пакет для построения графики разной степени сложности. В обычных текстах обычно не нужен

\usepackage{tikz}
\usepackage{smartdiagram}
\usetikzlibrary{positioning}
\usetikzlibrary{decorations} 
\usetikzlibrary{shapes,arrows}
\usepgflibrary{arrows.meta}
\usetikzlibrary{arrows.meta}
\usetikzlibrary{bending}
\usetikzlibrary{graphs}
\usetikzlibrary{decorations.pathmorphing}
\usetikzlibrary{calc,patterns,angles,quotes}


\renewcommand{\epsilon}{\ensuremath{\varepsilon}}
\renewcommand{\phi}{\ensuremath{\varphi}}
\renewcommand{\kappa}{\ensuremath{\varkappa}}
\renewcommand{\le}{\ensuremath{\leqslant}}
\renewcommand{\leq}{\ensuremath{\leqslant}}
\renewcommand{\ge}{\ensuremath{\geqslant}}
\renewcommand{\geq}{\ensuremath{\geqslant}}
\renewcommand{\emptyset}{\varnothing}


% Title
\UpperOrganization{МИНИСТЕРСТВО НАУКИ И ВЫСШЕГО ОБРАЗОВАНИЯ РОССИЙСКОЙ~ФЕДЕРАЦИИ}
\Organization{ФЕДЕРАЛЬНОЕ ГОСУДАРСТВЕННОЕ АВТОНОМНОЕ ОБРАЗОВАТЕЛЬНОЕ УЧРЕЖДЕНИЕ ВЫСШЕГО ОБРАЗОВАНИЯ	<<НОВОСИБИРСКИЙ НАЦИОНАЛЬНЫЙ ИССЛЕДОВАТЕЛЬСКИЙ ГОСУДАРСТВЕННЫЙ УНИВЕРСИТЕТ>>}
\Faculty{Физический факультет}
\Department{Кафедра физики ускорителей}
\title{Тема курсовой работы}
\author{ФИО автора}
\Group{12345}
\Institute{Институт}
\Laboratory{Лаборатория}
\Superviser{ФИО руководителя}
\KeyWords{электронный пучок, эмиттанс, ускоритель, уравнение огибающей}


\RefSource{bibliography}


\SetPDFmeta