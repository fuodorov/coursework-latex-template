\chapter{Первая глава}
Здесь текст первой главы. Пример с формулами.

\section{Уравнение огибающей для аксиально симметричного пучка в канале с соленоидальными линзами}
Движение аксиально-симметричного пучка в транспортном канале при наличии соленоидов может быть описано следующим уравнением~\ref{eq:envelope_axial}~\cite{Louson}:
\begin{equation}
    \label{eq:envelope_axial}
    \displaystyle r'' + \frac{1}{\beta^2\gamma} \gamma' r' + \frac{1}{2\beta^2\gamma}\gamma''r + k_sr - \frac{P}{r} - \frac{\epsilon^2}{r^3} = 0. 
\end{equation}
В уравнении рассматривается круглый пучок с радиусом $r$ и равномерным распределением плотности объемного заряда. В данном случае частицы запускаются с катода экранированного от магнитного поля (фактически это условие означает отсутствие углового момента $P_{\theta}$ = 0 ), $\beta$ "--- безразмерная скорость, $\gamma$ "--- Лоренц-фактор, $\gamma' = \dfrac{d\gamma}{dz}$, $\gamma'' = \dfrac{d^2\gamma}{dz^2}$,  $P = \dfrac{2I}{I_a\beta^3\gamma^3}$ "---обобщенный первеанс пучка, $I$ "--- ток пучка, $I_a = \dfrac{mc^3}{e} \approx 17$~кА, $\epsilon$ "--- эмиттанс пучка.
\[k_s =  \left ( \frac{eB_z}{2m_ec\beta\gamma} \right )^2 = \left ( \frac{e B_z}{2\beta\gamma\cdot 0.511\cdot 10^6 e \cdot \mathrm{volt}/c} \right )^2 =
\left ( \frac{cB_z[\mathrm{T}]}{2\beta\gamma\cdot 0.511\cdot 10^6 \cdot \mathrm{volt}} \right )^2\] --- жесткость соленоидальных линз. 
